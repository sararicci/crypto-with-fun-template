\documentclass{beamer}
\usetheme{Madrid}
\usecolortheme{default}

\title{Random Number Generators}
\subtitle{In-Class Exercise: Comparing Randomness}
\author{Cryptography Course}
\date{\today}

\begin{document}

\frame{\titlepage}

\begin{frame}
\frametitle{Exercise Overview}
\begin{block}{Goal}
Compare three different types of random number generators by generating and analyzing bit sequences.
\end{block}

\vspace{0.5cm}

\begin{itemize}
    \item \textbf{Duration:} 10--15 minutes
    \item \textbf{Task:} Generate three sequences of 100 bits
    \item \textbf{Analysis:} Visual inspection and simple randomness tests
\end{itemize}
\end{frame}

\begin{frame}
\frametitle{The Three Generators}
\begin{enumerate}
    \item \textbf{True Random Number Generator (TRNG)}
    \begin{itemize}
        \item Uses physical phenomena (atmospheric noise)
        \item Source: \texttt{random.org}
    \end{itemize}
    
    \vspace{0.3cm}
    
    \item \textbf{Linear Congruential Generator (LCG)}
    \begin{itemize}
        \item Deterministic algorithm: $X_{n+1} = (aX_n + c) \mod m$
        \item Fast but predictable
    \end{itemize}
    
    \vspace{0.3cm}
    
    \item \textbf{Cryptographically Secure PRNG (CSPRNG)}
    \begin{itemize}
        \item Designed for cryptographic applications
        \item Unpredictable even with partial knowledge
    \end{itemize}
\end{enumerate}
\end{frame}

\begin{frame}
\frametitle{Step 1: True RNG (random.org)}
\begin{enumerate}
    \item Visit: \texttt{https://www.random.org/integers/}
    \item Configure the generator:
    \begin{itemize}
        \item Number of integers: \textbf{100}
        \item Range: \textbf{0} to \textbf{1}
        \item Format: \textbf{One per line}
    \end{itemize}
    \item Click \textbf{``Get Numbers''}
    \item Copy the sequence and save it as \texttt{trng\_sequence.txt}
\end{enumerate}

\vspace{0.3cm}
\begin{alertblock}{Note}
This uses atmospheric noise for true randomness!
\end{alertblock}
\end{frame}

\begin{frame}[fragile]
\frametitle{Step 2: Linear Congruential Generator}
\begin{block}{Simple Python Implementation}
\begin{verbatim}
a, c, m = 1103515245, 12345, 2**31
seed = 42
sequence = []
for _ in range(100):
    seed = (a * seed + c) % m
    sequence.append(seed % 2)
print(''.join(map(str, sequence)))
\end{verbatim}
\end{block}

\vspace{0.2cm}
Save the output as \texttt{lcg\_sequence.txt}
\end{frame}

\begin{frame}[fragile]
\frametitle{Step 3: Cryptographically Secure PRNG}
\begin{block}{Python's \texttt{secrets} Module}
\begin{verbatim}
import secrets
sequence = [secrets.randbelow(2) 
            for _ in range(100)]
print(''.join(map(str, sequence)))
\end{verbatim}
\end{block}

\vspace{0.2cm}
Save the output as \texttt{csprng\_sequence.txt}

\vspace{0.3cm}
\begin{alertblock}{Security Note}
The \texttt{secrets} module is suitable for security-sensitive applications.
\end{alertblock}
\end{frame}

\begin{frame}
\frametitle{Step 4: Visual Analysis}
\begin{block}{Simple Tests to Perform}
\begin{enumerate}
    \item \textbf{Count 0s and 1s}
    \begin{itemize}
        \item Should be approximately 50/50
    \end{itemize}
    
    \item \textbf{Look for patterns}
    \begin{itemize}
        \item Long runs of same bit?
        \item Repetitive sequences?
    \end{itemize}
    
    \item \textbf{Compare sequences}
    \begin{itemize}
        \item Which one ``looks'' more random?
        \item Which one has unexpected patterns?
    \end{itemize}
\end{enumerate}
\end{block}
\end{frame}

\begin{frame}
\frametitle{Discussion Questions}
\begin{itemize}
    \item Can you visually distinguish between the three sequences?
    \vspace{0.3cm}
    
    \item What makes a sequence ``look random''?
    \vspace{0.3cm}
    
    \item Why might the LCG show patterns?
    \vspace{0.3cm}
    
    \item When would you use each type of generator?
    \vspace{0.3cm}
    
    \item What are the security implications of using weak RNGs in cryptography?
\end{itemize}
\end{frame}

\begin{frame}
\frametitle{Key Takeaways}
\begin{block}{Important Points}
\begin{itemize}
    \item \textbf{TRNGs} use physical randomness (unpredictable)
    \item \textbf{LCGs} are fast but unsuitable for cryptography
    \item \textbf{CSPRNGs} balance speed and security
    \item Visual inspection is \textit{not sufficient} for security
    \item Proper statistical tests are needed for cryptographic use
\end{itemize}
\end{block}

\vspace{0.5cm}
\centering
\Large Questions?
\end{frame}

\end{document}